% Template for Elsevier CRC journal article
% version 1.2 dated 09 May 2011

% This file (c) 2009-2011 Elsevier Ltd.  Modifications may be freely made,
% provided the edited file is saved under a different name

% This file contains modifications for Procedia Engineering

% Changes since version 1.1
% - added "procedia" option compliant with ecrc.sty version 1.2a
%   (makes the layout approximately the same as the Word CRC template)
% - added example for generating copyright line in abstract

%-----------------------------------------------------------------------------------

%% This template uses the elsarticle.cls document class and the extension package ecrc.sty
%% For full documentation on usage of elsarticle.cls, consult the documentation "elsdoc.pdf"
%% Further resources available at http://www.elsevier.com/latex

%-----------------------------------------------------------------------------------

%%%%%%%%%%%%%%%%%%%%%%%%%%%%%%%%%%%%%%%%%%%%%%%%%%%%%%%%%%%%%%
%%%%%%%%%%%%%%%%%%%%%%%%%%%%%%%%%%%%%%%%%%%%%%%%%%%%%%%%%%%%%%
%%                                                          %%
%% Important note on usage                                  %%
%% -----------------------                                  %%
%% This file should normally be compiled with PDFLaTeX      %%
%% Using standard LaTeX should work but may produce clashes %%
%%                                                          %%
%%%%%%%%%%%%%%%%%%%%%%%%%%%%%%%%%%%%%%%%%%%%%%%%%%%%%%%%%%%%%%
%%%%%%%%%%%%%%%%%%%%%%%%%%%%%%%%%%%%%%%%%%%%%%%%%%%%%%%%%%%%%%

%% The '3p' and 'times' class options of elsarticle are used for Elsevier CRC
%% The 'procedia' option causes ecrc to approximate to the Word template
\documentclass[3p,times,procedia,number]{elsarticle}
\flushbottom

\usepackage{color}

%% The `ecrc' package must be called to make the CRC functionality available
\usepackage{ecrc}
\usepackage{amsmath}


%% The ecrc package defines commands needed for running heads and logos.
%% For running heads, you can set the journal name, the volume, the starting page and the authors

%% set the volume if you know. Otherwise `00'
\volume{00}

%% set the starting page if not 1
\firstpage{1}

%% Give the name of the journal
\journalname{Procedia Engineering}

%% Give the author list to appear in the running head
%% Example \runauth{C.V. Radhakrishnan et al.}
\runauth{UGAWG}

%% The choice of journal logo is determined by the \jid and \jnltitlelogo commands.
%% A user-supplied logo with the name <\jid>logo.pdf will be inserted if present.
%% e.g. if \jid{yspmi} the system will look for a file yspmilogo.pdf
%% Otherwise the content of \jnltitlelogo will be set between horizontal lines as a default logo

%% Give the abbreviation of the Journal.
\jid{proeng}

%% Give a short journal name for the dummy logo (if needed)
%\jnltitlelogo{Procedia Engineering}

%% Hereafter the template follows `elsarticle'.
%% For more details see the existing template files elsarticle-template-harv.tex and elsarticle-template-num.tex.

%% Elsevier CRC generally uses a numbered reference style
%% For this, the conventions of elsarticle-template-num.tex should be followed (included below)
%% If using BibTeX, use the style file elsarticle-num.bst

%% End of ecrc-specific commands
%%%%%%%%%%%%%%%%%%%%%%%%%%%%%%%%%%%%%%%%%%%%%%%%%%%%%%%%%%%%%%%%%%%%%%%%%%

%% The amssymb package provides various useful mathematical symbols

\usepackage{amssymb}
%% The amsthm package provides extended theorem environments
%% \usepackage{amsthm}

%% The lineno packages adds line numbers. Start line numbering with
%% \begin{linenumbers}, end it with \end{linenumbers}. Or switch it on
%% for the whole article with \linenumbers after \end{frontmatter}.
%% \usepackage{lineno}

%% natbib.sty is loaded by default. However, natbib options can be
%% provided with \biboptions{...} command. Following options are
%% valid:

%%   round  -  round parentheses are used (default)
%%   square -  square brackets are used   [option]
%%   curly  -  curly braces are used      {option}
%%   angle  -  angle brackets are used    <option>
%%   semicolon  -  multiple citations separated by semi-colon
%%   colon  - same as semicolon, an earlier confusion
%%   comma  -  separated by comma
%%   numbers-  selects numerical citations
%%   super  -  numerical citations as superscripts
%%   sort   -  sorts multiple citations according to order in ref. list
%%   sort&compress   -  like sort, but also compresses numerical citations
%%   compress - compresses without sorting
%%
%\biboptions{authoryear}

 \biboptions{sort&compress}

% if you have landscape tables
\usepackage[figuresright]{rotating}
%\usepackage{harvard}
% put your own definitions here:x
%   \newcommand{\cZ}{\cal{Z}}
%   \newtheorem{def}{Definition}[section]
%   ...

% add words to TeX's hyphenation exception list
%\hyphenation{author another created financial paper re-commend-ed Post-Script}

% declarations for front matter

\begin{document}

\begin{frontmatter}

%% Title, authors and addresses

%% use the tnoteref command within \title for footnotes;
%% use the tnotetext command for the associated footnote;
%% use the fnref command within \author or \address for footnotes;
%% use the fntext command for the associated footnote;
%% use the corref command within \author for corresponding author footnotes;
%% use the cortext command for the associated footnote;
%% use the ead command for the email address,
%% and the form \ead[url] for the home page:
%%
%% \title{Title\tnoteref{label1}}
%% \tnotetext[label1]{}
%% \author{Name\corref{cor1}\fnref{label2}}
%% \ead{email address}
%% \ead[url]{home page}
%% \fntext[label2]{}
%% \cortext[cor1]{}
%% \address{Address\fnref{label3}}
%% \fntext[label3]{}

\dochead{25th International Meshing Roundtable}
%% Use \dochead if there is an article header, e.g. \dochead{Short communication}
%% \dochead can also be used to include a conference title, if directed by the editors
%% e.g. \dochead{17th International Conference on Dynamical Processes in Excited States of Solids}

\title{First benchmark of the Unstructued Grid Adaptation Working Group}

%% use optional labels to link authors explicitly to addresses:
%% \author[label1,label2]{<author name>}
%% \address[label1]{<address>}
%% \address[label2]{<address>}



\author[a]{Mike Park\corref{cor1}}
\author[b]{Nico Barral}
\author[b]{Adrien Loseille}
\author[c]{Daniel Ibanez}
\author[d]{Joshua Krakos}
\author[d]{Todd Michal}

\address[a]{NASA}
\address[b]{INRIA}
\address[c]{Sandia National Laboratories, P.O. Box 5800, Albuquerque, NM 87185-1321, United States}
\address[d]{Boeing}

\begin{abstract}
%% Text of abstract
Describe what we're trying to do, the website, and analyze the initial results.
\end{abstract}

\begin{keyword}
Type your keywords here, separated by semicolons ;

%% keywords here, in the form: keyword \sep keyword

%% PACS codes here, in the form: \PACS code \sep code

%% MSC codes here, in the form: \MSC code \sep code
%% or \MSC[2008] code \sep code (2000 is the default)

\end{keyword}
\cortext[cor1]{Corresponding author. Tel.: +0-000-000-0000 ; fax: +0-000-000-0000.}
\end{frontmatter}

%\correspondingauthor[*]{Corresponding author. Tel.: +0-000-000-0000 ; fax: +0-000-000-0000.}
\email{author@institute.xxx}

%%
%% Start line numbering here if you want
%%
% \linenumbers

%% main text

%\enlargethispage{-7mm}
\section{Introduction}
{\color{red} Mike}

Reference related things like the AIAA workshops

This paper describes an effort to benchmark
anisotropic grid adaptation tools by comparing different implementations
to understand the implications of implementation choices.
Then each tool can be verified
by comparison to alternative implementations or
an analytic result when available.
Descriptions of each implementation and the
results of this comparison are documented to ensure correct implementation
and set the stage for further method development.
This verification by comparison approach is also employed by the
Turbulence Modeling Resource Website.\cite{rumsey-smith-huang-turbmodels-description}
``What makes the current website unique is that it focuses on
providing ready access to equations, grids, and flow solution details
from previously verified codes as an aid to users
who wish to verify their own implementations of models on
relatively simple cases.''\cite{rumsey-smith-huang-turbmodels-description}
The goal of this work is to define a framework for
rigorous examination of anisotropic grid adaptation methods
that can guide the implementation and
further development of solution adaptive methods.

\section{Benchmarks Site}
{\color{red} Josh}

Website layout, case descriptions, evaluation framework.
Motivations for choosing certain metrics.

\section{Participating Codes}
{\color{red} Everyone}

\subsection{Omega\_h}

Omega\_h is an open-source grid adaptation library, written in C++11.
Like the other codes in this study, it aims to be a state-of-the-art
implementation of grid adaptation by local topological modifications.
Omega\_h has certain unique objectives.
First, it targets tightly coupled adaptivity within a simulation,
which requires remapping the solution accurately.
This motivates minimizing the number of modifications.
Second, it targets simulations outside the CFD space, including
solid mechanics and shock hydrodynamics.
This motivates a much stronger focus on element quality
and efficient operation with isotropic metrics.
Third, it targets high performance execution using threading
and even GPUs.

\section{Results}
{\color{red} Dan}

We first present the results from all
the currently participating codes using the linearly varying
metric on the unit cube.
For this publication, we focus on two measures of metric
satisfaction.
First, the classic edge-length criterion as presented in
(CITE THE LAST COMPARISON PAPER).
Second, we also present element quality results, using the
mean ratio formula for tetrahedra.

\begin{equation}
Q_K = \frac{\left(\frac{|K|\det(M)^{\frac12}}{|\hat{K}|}\right)^{\frac{2}{3}}}{\frac16\sum_{e\in K}{v_e^T M v_e}}
\end{equation}

%\enlargethispage{12pt}

\section{Future Directions}
{\color{red} Adrien}

\section{Conclusion}

\bibliography{references}
\bibliographystyle{model1-num-names}

\end{document}

\clearpage

%%%% This page is for instructions only, once the article is finalize please omit the below text before creating the final PDF
\normalMode

\section*{Instructions to Authors for LaTeX template:}

\section{ZIP mode for LaTeX template:}

The zip package is created as per the guide lines present on the URL http://www.elsevier.com/author-schemas/ preparing-crc-journal-articles-with-latex for creating the LaTeX zip file of Procedia LaTeX template.  The zip generally contains the following files:
\begin{Itemize}[]\leftskip-17.7pt\labelsep3.3pt
\item ecrc.sty
\item  elsarticle.cls
\item elsdoc.pdf
\item .bst file
\item Manuscript templates for use with these bibliographic styles
\item  Generic and journal specific logos, etc.
\end{Itemize}

The LaTeX package is the main LaTeX template. All LaTeX support files are required for LaTeX pdf generation from the LaTeX template package.

{\bf Reference style .bst file used for collaboration support:} In the LaTeX template packages of all Procedia titles a new ``.bst'' file is used which supports collaborations downloaded from the path http://www.elsevier.com/author-schemas/the-elsarticle-latex-document-class

\section{Reference styles used in  Procedia master templates:}
\let\footnotesize\normalsize
\hspace*{-10pt}\begin{tabular*}{\hsize}{@{}ll@{}}
{\bf Title}&{\bf Reference style} \\[6pt]
AASPRO  & 2 Harvard\\
AASRI Procedia  & 3 Vancouver Numbered\\
APCBEE Procedia  & 3 Vancouver Numbered\\
EGYPRO  & 3 Vancouver Numbered\\
FINE    & 2 Harvard\\
IERI Procedia  & 3 Vancouver Numbered\\
MATPR  & 1a Numbered without article titles\\
MSPRO  & 2 Harvard\\
PHPRO  & 2 Harvard\\
PIUTAM  & 3a Embellished Vancouver \\
Procedia CIRP  & 3 Vancouver Numbered\\
PROCHE  & 3a Embellished Vancouver \\
PROCS  & 3a Embellished Vancouver \\
PROENG  & 1 Numbered\\
PROENV  & 3a Embellished Vancouver \\
PROEPS  & 3a Embellished Vancouver \\
PROFOO    & 3a Embellished Vancouver \\
PROMFG  & 1a Numbered without article titles\\
PROTCY  & 3 Vancouver Numbered\\
PROVAC  & 3a Embellished Vancouver \\
SBSPRO  & 5 APA\\
SEPRO  & 3a Embellished Vancouver \\
AQPRO & 2 Harvard\\
UMKPRO & 5 APA\\
\end{tabular*}

\end{document}

%%
%% End of file `procs-template.tex'.
